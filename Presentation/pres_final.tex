\documentclass{beamer}
\usetheme{Copenhagen}
\usepackage{flexiprogram}
\usepackage{wrapfig}
\usepackage[usenames,dvipsnames]{pstricks}
\usepackage{epsfig}
\usepackage{pst-func}
\usepackage{pst-grad} % For gradients
\usepackage{pst-plot} % For axes
\usepackage{pst-node}
\usepackage{url}
\usepackage{ulem}
%\usepackage{enumerate}

% \usepackage[labelfont={bf}]{caption}

\bibliographystyle{plain}

\title{Greenhouse Monitoring (And Harvesting) Using Accurate And Automated Bot Guidance System}
\author{Devendra Bhave (114050004) \\ \and Mohd Vasimuddin (114050007) \\ \and Meenakshi Verma (123050014) \\ \and Mukund Lahoti (123050018)}
\date{\today}

\setbeamertemplate{footline}[frame number]

\begin{document}

\frame{\titlepage}

\frame{\frametitle{Contents}\tableofcontents} 

\section{Problem Statement}

\frame{\frametitle{Problem Statement}
 \structure{Develop completely automatic greenhouse harvesting and monitoring system using FireBird platform.}
 
 \vspace{\baselineskip}
 Subgoals:
 \begin{itemize}
  \item Build automatic bot guidance system
  \item Use OpenCV to process images from IP camera
  \item Control Bot over Zigbee for automatic harvesting and monitoring
 \end{itemize}
} 

\frame{\frametitle{Functional Requirements}
Automated Bot Guidance System:
\begin{enumerate}
 \item Maneuver the Bot automatically
 \item Support \alert{\texttt{goto(x, y)}} primitive
 \item {\color{rgb:green,2;black,1} Guarantee accuracy}
\end{enumerate}
}

\frame{\frametitle{Functional Requirements}
Greenhouse Management:
 \only<1> {
\begin{enumerate}
 \item \alert{Semi automatic greenhouse monitoring}
 \item \alert{Semi automatic harvesting}
 \end{enumerate}
 }
 
 \only<2> {
 \begin{enumerate}
 \item \alert{\sout{Semi} automatic greenhouse monitoring}
 \item \alert{Semi automatic harvesting}
 \end{enumerate}
 }
 
 \only<3> {
 \begin{enumerate}
 \item \alert{\sout{Semi}} {\color{rgb:green,2;black,1} automatic greenhouse monitoring}
 \item \alert{Semi automatic harvesting}
 \end{enumerate}
 }
 
 \only<4> {
 \begin{enumerate}
 \item \alert{\sout{Semi}} {\color{rgb:green,2;black,1} automatic greenhouse monitoring}
 \item \alert{\sout{Semi} automatic harvesting}
 \end{enumerate}
 }
 
 \only<5> {
 \begin{enumerate}
 \item \alert{\sout{Semi}} {\color{rgb:green,2;black,1} automatic greenhouse monitoring}
 \item \alert{\sout{Semi}} {\color{rgb:green,2;black,1} automatic harvesting}
 \end{enumerate}
 }
}

\section{System Description}


\frame{\frametitle{Bot Description}
% Generated with LaTeXDraw 2.0.8
% Thu Nov 08 00:11:11 IST 2012
% \usepackage[usenames,dvipsnames]{pstricks}
% \usepackage{epsfig}
% \usepackage{pst-grad} % For gradients
% \usepackage{pst-plot} % For axes
\scalebox{0.7} % Change this value to rescale the drawing.
{
\begin{pspicture}(0,-4.92)(15.8,4.9)
\psframe[linewidth=0.04,dimen=outer](3.6,-0.3)(0.0,-2.5)
\psframe[linewidth=0.04,dimen=outer](5.6,0.7)(2.0,-1.5)
\psarc[linewidth=0.04](3.8,0.7){1.8}{0.0}{180.0}
\psarc[linewidth=0.04](1.8,-0.3){1.8}{0.0}{180.0}
\psline[linewidth=0.04cm](3.6,-2.5)(5.6,-1.5)
\psline[linewidth=0.04cm](3.6,-0.3)(5.6,0.7)
\psline[linewidth=0.04cm](1.0,1.3)(3.0,2.3)
\psline[linewidth=0.04cm](0.0,-2.5)(2.0,-1.5)
\psline[linewidth=0.04cm](0.0,-0.3)(2.0,0.7)
\psline[linewidth=0.04cm](1.8,-1.9)(2.6,-1.9)
\psline[linewidth=0.04cm](1.8,-1.9)(1.8,-2.3)
\psline[linewidth=0.04cm](2.6,-1.9)(2.6,-2.3)
\psline[linewidth=0.04cm](1.8,-2.3)(2.6,-2.3)
\psline[linewidth=0.04cm](1.8,-1.9)(2.2,-1.7)
\psline[linewidth=0.04cm](2.2,-1.7)(3.0,-1.7)
\psline[linewidth=0.04cm](3.0,-1.7)(2.6,-1.9)
\psline[linewidth=0.04cm](3.0,-1.7)(3.0,-2.1)
\psline[linewidth=0.04cm](2.6,-2.3)(3.0,-2.1)
\pscircle[linewidth=0.04,linestyle=dashed,dash=0.16cm 0.16cm,dimen=outer](2.4,-2.1){0.8}
\psline[linewidth=0.04cm](7.0,-0.9)(7.6,-0.3)
\psline[linewidth=0.04cm](11.6,-0.3)(11.0,-0.9)
\psline[linewidth=0.04cm](11.6,-0.3)(11.6,-2.1)
\psline[linewidth=0.04cm](11.6,-2.1)(11.0,-2.7)
\pscircle[linewidth=0.04,dimen=outer](9.1,-2.6){1.1}
\psline[linewidth=0.04cm](10.2,-2.7)(11.0,-2.7)
\psline[linewidth=0.04cm](11.0,-2.7)(11.0,-0.9)
\psline[linewidth=0.04cm](11.0,-0.9)(7.0,-0.9)
\psline[linewidth=0.04cm](7.0,-0.9)(7.0,-2.7)
\psline[linewidth=0.04cm](7.0,-2.7)(8.0,-2.7)
\psarc[linewidth=0.04](9.9,-2.0){1.1}{-80.0}{-40.0}
\psline[linewidth=0.04cm](8.2,2.7)(8.2,0.3)
\psline[linewidth=0.04cm](8.2,2.7)(8.8,3.3)
\psline[linewidth=0.04cm](8.8,3.3)(8.8,0.9)
\psline[linewidth=0.04cm](8.8,0.9)(8.2,0.3)
\psline[linewidth=0.04cm](8.2,2.7)(7.8,2.7)
\psline[linewidth=0.04cm](7.8,2.7)(7.8,0.3)
\psline[linewidth=0.04cm](7.8,2.7)(8.4,3.3)
\psline[linewidth=0.04cm](8.4,3.3)(8.8,3.3)
\psline[linewidth=0.04cm](9.8,0.1)(9.8,-0.7)
\psline[linewidth=0.04cm](10.4,0.1)(10.4,-0.7)
\psline[linewidth=0.04cm](9.8,-0.7)(10.4,-0.7)
\psline[linewidth=0.04cm](10.6,-0.1)(10.6,-0.5)
\psline[linewidth=0.04cm](10.6,-0.5)(10.4,-0.7)
\psline[linewidth=0.04cm](10.6,0.1)(10.6,-0.1)
\psline[linewidth=0.04cm](11.6,-0.3)(10.6,-0.3)
\psline[linewidth=0.04cm](9.8,-0.3)(7.6,-0.3)
\psellipse[linewidth=0.04,dimen=outer](8.5,2.5)(0.1,0.2)
\pscircle[linewidth=0.04,linestyle=dashed,dash=0.16cm 0.16cm,dimen=outer](10.9,0.0){4.9}
\psline[linewidth=0.04cm,linestyle=dashed,dash=0.16cm 0.16cm](2.2,-1.3)(8.2,4.1)
\psline[linewidth=0.04cm,linestyle=dashed,dash=0.16cm 0.16cm](2.2,-2.9)(10.2,-4.9)
\usefont{T1}{ptm}{m}{n}
\rput(9.697344,-1.195){FireBird Bot}
\usefont{T1}{ptm}{m}{n}
\rput(1.6484375,2.405){Greenhouse}
\usefont{T1}{ptm}{m}{n}
\rput(11.926094,4.005){Robotic Arm}
\psline[linewidth=0.04cm,arrowsize=0.05291667cm 2.0,arrowlength=1.4,arrowinset=0.4]{->}(11.8,3.7)(11.8,1.7)
\psline[linewidth=0.04cm](7.6,0.3)(12.0,0.3)
\psline[linewidth=0.04cm](8.8,0.9)(11.4,0.9)
\psline[linewidth=0.04cm](11.8,0.9)(12.4,0.9)
\psline[linewidth=0.04cm](12.4,0.9)(12.0,0.3)
\psline[linewidth=0.04cm](7.6,0.3)(7.8,0.5)
\psline[linewidth=0.04cm](7.6,0.3)(7.6,0.1)
\psline[linewidth=0.04cm](7.6,0.1)(12.0,0.1)
\psline[linewidth=0.04cm](12.0,0.3)(12.0,0.1)
\psline[linewidth=0.04cm](12.4,0.9)(12.4,0.7)
\psline[linewidth=0.04cm](12.4,0.7)(12.0,0.1)
\pspolygon[linewidth=0.04,fillstyle=solid](11.4,0.7)(12.6,2.9)(12.8,2.9)(11.8,0.7)
\pspolygon[linewidth=0.04,fillstyle=solid](11.2,0.5)(12.4,2.7)(12.6,2.7)(11.6,0.5)
\psframe[linewidth=0.04,dimen=outer,fillstyle=solid](13.0,2.9)(12.2,2.7)
\pspolygon[linewidth=0.04](12.6,2.9)(13.6,3.5)(13.0,2.9)
\pspolygon[linewidth=0.04](13.0,2.9)(14.0,3.3)(13.0,2.7)
\usefont{T1}{ptm}{m}{n}
\rput(14.107187,1.405){Cutters}
\usefont{T1}{ptm}{m}{n}
\rput(9.575,4.005){IP Camera}
\psline[linewidth=0.04cm,arrowsize=0.05291667cm 2.0,arrowlength=1.4,arrowinset=0.4]{->}(14.0,1.7)(13.4,2.7)
\psline[linewidth=0.04cm,arrowsize=0.05291667cm 2.0,arrowlength=1.4,arrowinset=0.4]{->}(9.6,3.7)(9.0,3.3)
\psframe[linewidth=0.04,dimen=outer,fillstyle=solid](13.6,-1.3)(11.0,-2.7)
\psline[linewidth=0.04cm](14.2,-0.7)(13.6,-1.3)
\psline[linewidth=0.04cm](14.2,-2.1)(13.6,-2.7)
\psline[linewidth=0.04cm](11.6,-0.7)(14.2,-0.7)
\psline[linewidth=0.04cm](14.2,-0.7)(14.2,-2.1)
\usefont{T1}{ptm}{m}{n}
\rput(14.438125,0.605){Fruit Collector}
\psline[linewidth=0.04cm,arrowsize=0.05291667cm 2.0,arrowlength=1.4,arrowinset=0.4]{->}(14.2,0.3)(13.2,-0.7)
\end{pspicture} 
}

}

\frame{\frametitle{System Architecture}
 % Generated with LaTeXDraw 2.0.8
% Thu Nov 08 00:26:46 IST 2012
% \usepackage[usenames,dvipsnames]{pstricks}
% \usepackage{epsfig}
% \usepackage{pst-grad} % For gradients
% \usepackage{pst-plot} % For axes
\scalebox{0.7} % Change this value to rescale the drawing.
{
\begin{pspicture}(0,-2.5)(15.780625,2.5)
\usefont{T1}{ptm}{m}{n}
\rput(5.1175,-1.935){Firebird}
\usefont{T1}{ptm}{m}{n}
\rput(0.76234376,-0.995){ADC}
\usefont{T1}{ptm}{m}{n}
\rput(2.108125,-0.995){Buzzer}
\usefont{T1}{ptm}{m}{n}
\rput(3.5495312,-0.995){LCD}
\usefont{T1}{ptm}{m}{n}
\rput(4.848125,-0.995){Motor}
\usefont{T1}{ptm}{m}{n}
\rput(6.268125,-0.995){Power}
\usefont{T1}{ptm}{m}{n}
\rput(7.6025,-0.995){Servo}
\usefont{T1}{ptm}{m}{n}
\rput(9.0725,-0.995){Zigbee}
\psframe[linewidth=0.04,dimen=outer](9.8,-0.5)(0.0,-1.5)
\psline[linewidth=0.04cm](8.4,-0.5)(8.4,-1.5)
\psline[linewidth=0.04cm](7.0,-0.5)(7.0,-1.5)
\psline[linewidth=0.04cm](5.6,-0.5)(5.6,-1.5)
\psline[linewidth=0.04cm](4.2,-0.5)(4.2,-1.5)
\psline[linewidth=0.04cm](2.8,-0.5)(2.8,-1.5)
\psline[linewidth=0.04cm](1.4,-0.5)(1.4,-1.5)
\psframe[linewidth=0.04,dimen=outer](9.8,-1.5)(0.0,-2.5)
\psframe[linewidth=0.04,dimen=outer](9.8,0.5)(0.0,-0.5)
\psline[linewidth=0.04cm](5.0,-0.5)(5.0,0.5)
\usefont{T1}{ptm}{m}{n}
\rput(2.5323439,0.005){Assertions}
\usefont{T1}{ptm}{m}{n}
\rput(7.262656,0.005){AVR-libc}
\psframe[linewidth=0.04,dimen=outer](9.8,1.5)(0.0,0.5)
\psline[linewidth=0.04cm](5.0,1.5)(5.0,0.5)
\usefont{T1}{ptm}{m}{n}
\rput(2.4160938,1.005){ROM Filesystem}
\usefont{T1}{ptm}{m}{n}
\rput(7.55875,1.005){Whiteline Follower}
\psframe[linewidth=0.04,dimen=outer](9.8,2.5)(0.0,1.5)
\usefont{T1}{ptm}{m}{n}
\rput(4.96625,2.005){Bot Guidance System}
\usefont{T1}{ptm}{m}{n}
\rput(11.297344,-1.995){Hardware}
\usefont{T1}{ptm}{m}{n}
\rput(13.165,-0.995){Hardware Abstraction Layer (HAL)}
\usefont{T1}{ptm}{m}{n}
\rput(11.891875,0.005){Software Libraries}
\usefont{T1}{ptm}{m}{n}
\rput(11.1328125,1.005){Utilities}
\usefont{T1}{ptm}{m}{n}
\rput(11.406875,2.005){Application}
\end{pspicture} 
}

}

\frame{\frametitle{Hardware Abstraction Layer (HAL)}
 Offers functionality specific interfaces
 
 \vspace{\baselineskip}
 
 Provides \texttt{init}$<$ModuleName$>$\texttt{()} interface for module initialization 
 
}

\frame{\frametitle{HAL -- ADC, Buzzer}
Interfaces:

\begin{itemize}
 \item \texttt{initAdc()}
 \begin{itemize}
  \item Initialize ADC hardware
 \end{itemize}
 \item \texttt{getAdcValue(adc\_channel)}
 \begin{itemize}
  \item Read ADC value for specified channel
 \end{itemize} 
\end{itemize}

\vspace{\baselineskip}
\begin{itemize}
 \item \texttt{initBuzzer()}
 \begin{itemize}
  \item Initialize ADC hardware
 \end{itemize}
 \item \texttt{buzzerOn()}
 \begin{itemize}
  \item Turn on buzzer
 \end{itemize}
 \item \texttt{buzzerOff()}
 \begin{itemize}
  \item Turn off buzzer
 \end{itemize} 
\end{itemize} 
}

\frame{\frametitle{HAL -- LCD}
Interfaces:

\begin{itemize}
 \item \texttt{initLcd()}
 \begin{itemize}
  \item Initialize LCD hardware
 \end{itemize}
 \item \texttt{lcdHome()}
 \begin{itemize}
  \item Place cursor at first column on LCD display
 \end{itemize}
 \item \texttt{lcdClear()}
 \begin{itemize}
  \item Clear LCD screen
 \end{itemize} 
 \item \texttt{lcdCursor(row, column)}
 \begin{itemize}
  \item Place cursor at given row and column on LCD display
 \end{itemize} 
 \item \texttt{lcdString(data)}
 \begin{itemize}
  \item Write data on LCD display
 \end{itemize} 
\end{itemize}
}

\frame{\frametitle{HAL -- Motor}
\begin{itemize}
 \item \texttt{initMotor()}
 \begin{itemize}
  \item Initialize DC motor hardware
 \end{itemize}
 \item \texttt{motorDirectionSet(direction)}
 \begin{itemize}
  \item Controls DC motors for direction
 \end{itemize}
 \item \texttt{motorVelocitySet(left\_vel, right\_vel)}
 \begin{itemize}
  \item Set motor velocity using PWM
 \end{itemize} 
 \item \texttt{motorVelocityGet()}
 \begin{itemize}
  \item Read motor velocity settings
 \end{itemize} 
 \item \texttt{motorLeftPositionEncoderInit(lCallback)}
 \begin{itemize}
  \item Register left positional encoder callback
 \end{itemize} 
 \item \texttt{motorRightPositionEncoderInit(rCallback)}
 \begin{itemize}
  \item Register right positional encoder callback
 \end{itemize}
 \item \texttt{motorLeftPositionEncoderInterruptConfig(state)}
 \begin{itemize}
  \item Enable/disable left positional encoder interrupt
 \end{itemize}
 \item \texttt{motorRightPositionEncoderInterruptConfig(state)}
 \begin{itemize}
  \item Enable/disable right positional encoder interrupt
 \end{itemize}
 \end{itemize}
}

\frame{\frametitle{HAL -- Power}
Sensor groups:
\begin{enumerate}
 \item \texttt{SG\_GROUP1}: Sharp IR range sensors 2,3,4 and white line LEDs
 \item \texttt{SG\_GROUP2}: Sharp IR range sensors 1,5
 \item \texttt{SG\_GROUP3}: IR proximity sensors
\end{enumerate}

\vspace{\baselineskip}
\begin{itemize}
 \item \texttt{initPower()}
 \begin{itemize}
  \item Initialize power management hardware
 \end{itemize}
 \item \texttt{powerOn(sensor\_group)}
 \begin{itemize}
  \item Turns power on for given group of sensors
 \end{itemize}
 \item \texttt{powerOff(sensor\_group)}
 \begin{itemize}
  \item Turns power off for given group of sensors
 \end{itemize} 
\end{itemize} 
 }

 \frame{\frametitle{HAL -- Servo}
\begin{itemize}
 \item \texttt{initServo()}
 \begin{itemize}
  \item Initialize servo motor hardware
 \end{itemize}
 \item \texttt{servoSet(motor, angle)}
 \begin{itemize}
  \item Sets given angle for servo motor
 \end{itemize}
 \item \texttt{servoFree(motor)}
 \begin{itemize}
  \item Unlocks servo motors
 \end{itemize} 
\end{itemize} 
 }

\frame{\frametitle{Assertions}
 Offers facility on Firebird for assertion based debugging
 
 \vspace{\baselineskip}
 
 Use \texttt{ASSERT(condition)} in the code
 
 \vspace{\baselineskip}
 
 If condition fails
 \begin{itemize}
  \item bot halts and beeps continuously
  \item displays line number, filename and failed condition on LCD screen
 \end{itemize}
 
 \vspace{\baselineskip}
 
 Assertions are replaced by \structure{empty statements} when not debugging.
}

\frame{\frametitle{ROM Filesystem}
 No filesystem support in AVR libc. So, we \alert{wrote our own}.
 
 \vspace{\baselineskip}
 Map file for bot guidance system can be added at compile time.
 
 \vspace{\baselineskip}
 Map file can be accessed using file handle \texttt{MAP\_FILE} for \texttt{fscanf()},
 \texttt{fgets()}, etc.
 
 \vspace{\baselineskip}
 Standard file streams (STDIN, STDOUT) are redirected to Zigbee. \texttt{printf()}, \texttt{scanf()} used anywhere in
 code sends or receives data over Zigbee.
 
 \vspace{\baselineskip}
 \structure{Greenhouse map can be loaded at runtime.}
}

\frame{\frametitle{Whiteline Follower}
Supported primitives:

 \vspace{\baselineskip}
\begin{itemize}
 \item \texttt{moveForwardFollwingLineByDistance(distance)}
 \begin{itemize}
  \item Moves along whiteline till specified distance is covered
 \end{itemize} 
 \item \texttt{moveForwardFollwingLineByCheckpoint(exp\_dist)}
 \begin{itemize}
  \item Moves along whiteline until checkpoint is hit
 \end{itemize} 
 \item \texttt{rotateBot(direction, angle)}
 \begin{itemize}
  \item Rotates bot in specified direction by given degrees
 \end{itemize} 
\end{itemize} 

}

\frame{\frametitle{Checkpoint Synchronization Automaton}
% Generated with LaTeXDraw 2.0.8
% Thu Nov 08 01:30:12 IST 2012
% \usepackage[usenames,dvipsnames]{pstricks}
% \usepackage{epsfig}
% \usepackage{pst-grad} % For gradients
% \usepackage{pst-plot} % For axes
\scalebox{0.7} % Change this value to rescale the drawing.
{
\begin{pspicture}(0,-4.408125)(15.7890625,4.408125)
\usefont{T1}{ptm}{m}{n}
\rput(2.501875,1.4146875){\psframebox[linewidth=0.04]{Outside Checkpoint Range}}
\usefont{T1}{ptm}{m}{n}
\rput(10.635938,2.6146874){Within Checkpoint Range}
\usefont{T1}{ptm}{m}{n}
\rput(12.367344,-1.3853126){Checkpoint Missed}
\usefont{T1}{ptm}{m}{n}
\rput(2.7435937,-1.3853126){Unexpected Checkpoint Error}
\usefont{T1}{ptm}{m}{n}
\rput(7.8323436,-1.3853126){At Checkpoint}
\psarc[linewidth=0.04,arrowsize=0.05291667cm 2.0,arrowlength=1.4,arrowinset=0.4]{->}(3.0934374,1.6096874){0.7}{0.0}{180.0}
\psframe[linewidth=0.04,dimen=outer](12.593437,3.1096876)(8.593437,2.3096876)
\rput{-45.0}(0.6624435,8.418655){\psarc[linewidth=0.04,arrowsize=0.05291667cm 2.0,arrowlength=1.4,arrowinset=0.4]{->}(10.493438,3.4096875){0.5}{0.0}{270.0}}
\psframe[linewidth=0.04,dimen=outer](13.993438,-0.8903125)(10.793438,-1.6903125)
\psframe[linewidth=0.04,dimen=outer](8.993438,-0.8903125)(6.5934377,-1.6903125)
\psframe[linewidth=0.04,dimen=outer](4.9934373,-0.8903125)(0.3934375,-1.6903125)
\psline[linewidth=0.04cm,arrowsize=0.05291667cm 2.0,arrowlength=1.4,arrowinset=0.4]{->}(4.5934377,1.5096875)(8.593437,2.7096875)
\psline[linewidth=0.04cm,arrowsize=0.05291667cm 2.0,arrowlength=1.4,arrowinset=0.4]{->}(10.993438,2.3096876)(12.393437,-0.8903125)
\psline[linewidth=0.04cm,arrowsize=0.05291667cm 2.0,arrowlength=1.4,arrowinset=0.4]{->}(10.993438,2.3096876)(7.9934373,-0.8903125)
\psline[linewidth=0.04cm,arrowsize=0.05291667cm 2.0,arrowlength=1.4,arrowinset=0.4]{->}(2.3934374,1.1096874)(1.5934376,-0.8903125)
\usefont{T1}{ptm}{m}{n}
\rput(0.961875,-2.5853126){Legend:}
\usefont{T1}{ptm}{m}{n}
\rput(2.1284375,-2.9853125){D: Distance covered yet}
\usefont{T1}{ptm}{m}{n}
\rput(2.979375,-3.3853126){$R_L$: Checkpoint range lower bound}
\usefont{T1}{ptm}{m}{n}
\rput(3.009375,-3.7853124){$R_H$: Checkpoint range upper bound}
\usefont{T1}{ptm}{m}{n}
\rput(5.1403127,-4.1853123){C: Checkpoint sensing (1 means checkpoint hit, 0 otherwise)}
\usefont{T1}{ptm}{m}{n}
\rput(1.7079687,2.6146874){<$D < R_L$, $C==0$>}
\usefont{T1}{ptm}{m}{n}
\rput(6.2079687,2.8146875){<$D == R_L$, $C==0$>}
\usefont{T1}{ptm}{m}{n}
\rput(10.827969,4.2146873){<$D <= R_H$, $C==0$>}
\usefont{T1}{ptm}{m}{n}
\rput(13.967969,0.8146875){<$D > R_H$, $C==X$>}
\usefont{T1}{ptm}{m}{n}
\rput(1.7079687,0.2146875){\psframebox*[framesep=0, boxsep=false,fillcolor=white] {<$D < R_L$, $C==1$>}}
\usefont{T1}{ptm}{m}{n}
\rput(9.027968,1.0146875){\psframebox*[framesep=0, boxsep=false,fillcolor=white] {<$D <= R_H$, $C==1$>}}
\psline[linewidth=0.04cm,arrowsize=0.05291667cm 2.0,arrowlength=1.4,arrowinset=0.4]{->}(3.5934374,1.1096874)(6.9934373,-0.8903125)
\usefont{T1}{ptm}{m}{n}
\rput(6.007969,-0.1853125){\psframebox*[framesep=0, boxsep=false,fillcolor=white] {<$D == R_L$, $C==1$>}}
\end{pspicture} 
}

}

\frame{\frametitle{Bot Guidance System}
Design summary:
\begin{itemize}
 \item Map of arena is precisely known
 \item Use white line to follow shortest path to destination from current location
 \item Add checkpoints to mitigate location errors
\end{itemize}

\vspace{\baselineskip}
\pause
Supported primitives:
\begin{itemize}
 \item \texttt{gotoPosition(x, y)}
 \begin{itemize}
  \item x and y are destination co-ordinates in millimeters
 \end{itemize} 
 \item \texttt{setBotOrientation(orientation)}
 \begin{itemize}
  \item Changes bot orientation
 \end{itemize} 
\end{itemize}
 
}

\frame{\frametitle{Bot Guidance System}
Features:

\begin{itemize}
 \item Pre-computes all node shortest paths using Floyd-Warshall algorithm
 \item Uses integer only computations for speed
 \item Uses fast integer square roots
 \item Assertion based validation
 \item \alert{Loop invariants analysis and testing using assertions} -- first step towards \structure{formal verification}
 \item Exhaustive test procedures
\end{itemize}
}

\frame{\frametitle{Linux Applications}
Harvesting and monitoring tasks are simple Linux applications.

\vspace{\baselineskip}
\structure{Harvesting}:
\begin{itemize}
 \item Use \texttt{gotoPosition()} primitive to move to desired trough
 \item Use CURL to fetch image from IP camera
 \item Use OpenCV to process image and detect cutter and fruits
 \item Control Bot over Zigbee to adjust cutter and cut fruits
\end{itemize}

\vspace{\baselineskip}
\structure{Monitoring}:
\begin{itemize}
 \item Use \texttt{gotoPosition()} primitive to move to desired trough
 \item Use CURL to fetch images from IP camera
 \item Save images for user to view
\end{itemize}

}

\section{Energy Analysis}
\frame{\frametitle{Battery Levels}
With repeated experiments, we have defined following battery levels.

\vspace{\baselineskip}
\begin{tabular}{|c|c|l|}
 \hline
 Battery Voltage & Battery API value & Meaning\\
 \hline
 \hline
 9.0 V & $> 900$ & Battery is sufficiently charged.\\
 \hline
 8.0 V & $< 800$ & Battery is low\\
 & & Do not start new task.\\
 & & Finish current task.\\
 \hline
 7.5 V & $< 750$ &  Battery is critically low.\\
 & & Abandon current task.\\
 & & Turn off all servo motors\\
 & & Run towards recharge station.\\
 \hline
\end{tabular}

}

\frame{\frametitle{Energy Statistics}
\begin{tabular}{|l|l|l|}
 \hline
 Action & Energy (Watt-Sec) & Energy (Watt-Hr)\\
 \hline
 \hline
 Move Bot along whiteline & 358 per meter & 0.01 per meter\\
 \hline
 Cutter and arm movement & &\\
 + move forward by 10cm & 340 & 0.094\\
 \hline
 Scan sideways for fruits & 1326 & 0.368\\
 \hline
 Cutting fruit & 1029 per fruit & 0.286 per fruit\\
 \hline
 One trajectory & 2875 & 0.799\\
 \hline
 One trough & &  \\
 (= 5 trajectories)& 14375 & 3.99\\
 \hline
 \end{tabular}
 
}

\section{Challeges Faced}
\frame{\frametitle{Challenge \#1: Reliable communication over Zigbee}
\begin{itemize}
 \item You cannot directly send binary data as it is over Zigbee
 \item Non-ASCII characters are treated as modem control commands in Linux serial COM driver stack
 \item We spent lot of time over identifying exact cause
 \item We used encoding in ASCII when sending non-ASCII data over Zigbee
 \item Designed communication protocol that uses \alert{$!$} terminated ASCII packets for 
 sending data over Zigbee.
 \item Remember you cannot send NULL character over Zigbee!
\end{itemize}
}

\frame{\frametitle{Challenge \#2: Mitigating wheel slip}
\begin{itemize}
 \item Wheel slipping was major concern.
 \item Bot travels less distance than expected using positional encoders.
 \item Reducing motor speed impacts bot movement.
 \item Center of gravity towards rear wheels reduces wheel slipping.
\end{itemize}
}

\frame{\frametitle{Challenge \#3: Design of fruit cutter}
\begin{itemize}
 \item Which is the best design?
 \item Design should be generic enough for cutting wide variety of fruits
 \item Design should address reliability concerns
 \item Goal was to come up with \structure{simple and power efficient design}
\end{itemize}
}

\section{Future Scope}

\frame{\frametitle{Future scope}
\begin{enumerate}
 \item Improve Bot guidance system to work with complex, non-grid maps.
 \item Design method to retrieve depth information for fruit cutting.
 \item Complex natural arm movements with better cutting trajectory
\end{enumerate}


}

\frame{\frametitle{References}
\bibliography{sample}
}
\end{document}

